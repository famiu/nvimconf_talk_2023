%! TeX program = lualatex
\documentclass[10pt]{beamer}
\title{No other \lq{}option\rq{}}
\subtitle{Neovim option internals}
\author{Famiu Haque}
\date{Neovim Conf 2023}
\usepackage[T1]{fontenc}
\usepackage{fontspec}
\usepackage{parskip}
\usepackage{graphicx}
\usepackage{mathtools}
\usepackage{listings}
\usepackage{fontawesome5}
\usepackage{microtype}
\usetheme[numbering=fraction]{metropolis}
\usecolortheme{owl}

\definecolor{titlefg}{RGB}{114, 169, 207}
\definecolor{listingsbg}{RGB}{10, 10, 30}

\setmainfont{Fira Sans}[ Scale = 0.7, UprightFont = Fira Sans Light, BoldFont = Fira Sans Medium,
ItalicFont = Fira Sans Light Italic, BoldItalicFont = Fira Sans Medium Italic ]
\setsansfont{Fira Sans}[ Scale = 0.7, UprightFont = Fira Sans Light, BoldFont = Fira Sans Medium,
ItalicFont = Fira Sans Light Italic, BoldItalicFont = Fira Sans Medium Italic ]
\setmonofont{Fira Code}[ Scale = 0.6 ]
\setmathsf{Fira Math}[ Scale = 0.7 ]
\setbeamertemplate{frametitle}[default][center]
\setbeamertemplate{frame footer}{\insertshortauthor\hfill\insertshorttitle: \insertshortsubtitle}
\setbeamercolor{frametitle}{bg=,fg=titlefg}
\setbeamercolor{framesubtitle}{bg=,fg=white}
\setbeamerfont{frametitle}{size=\LARGE}
\setbeamerfont{framesubtitle}{size=\Large,series=\mdseries}
% Define
\lstdefinelanguage{vim}
{%
    morekeywords={
        set,
        setlocal,
        setglobal,
        let,
        map,
        nmap,
        filetype,
        on,
        off,
        autocmd,
        Plugin,
        call
    },
    morecomment=[l]{"}, % l is for line comment
    morestring=[b]' % defines that strings are enclosed in double quotes
}
% Default code style
\lstdefinestyle{code_default_style}{
    backgroundcolor    = \color{listingsbg},
    commentstyle       = \ttfamily\color{lightgray},
    keywordstyle       = \ttfamily\color{red},
    numberstyle        = \ttfamily\color{blue},
    stringstyle        = \ttfamily\color{green},
    basicstyle         = \ttfamily,
    frame = single,
    framesep = 5pt,
    xleftmargin = 5pt,
    xrightmargin = 5pt,
    aboveskip          = 10pt,
    belowskip          = 5pt,
    tabsize            = 4,
    showstringspaces   = false,
    upquote            = true,
    breakatwhitespace  = false,
    breaklines         = true,
    captionpos         = b,
    keepspaces         = true,
    showspaces         = false,
    showstringspaces   = false,
    showtabs           = false,
}
\lstdefinestyle{vim_style}{
    language           = vim,
    style              = code_default_style,
}
\lstdefinestyle{lua_style}{
    language           = {[5.1]Lua},
    style              = code_default_style,
}
\lstdefinestyle{c_style}{
    language           = C,
    style              = code_default_style,
}
\lstset{style=vim_style}

\begin{document}
\begin{frame}
\titlepage
\end{frame}
\begin{frame}
\frametitle{About Me}
\begin{itemize}
    \item Neovim contributor since 2021.
    \item Author of global statusline, winbar, user inccommand, `nvim\_cmd`, among other things.
    \item Currently working full-time to refactor Neovim options.
    \item Still a high-school student.
    \item Social media:
\end{itemize}
\vspace{0.5em}
\hspace{3em}
\begin{tabular}{l@{\hspace{0.5em}}l@{\hspace{8\tabcolsep}}l@{\hspace{0.5em}}l}
    \faIcon{github} & github.com/famiu & \faIcon{twitch} & twitch.tv/fam1u \\
    \faIcon{youtube} & youtube.com/@fam1u & \faIcon{twitter} & twitter.com/famiuhaque \\
\end{tabular}
\end{frame}
\begin{frame}
\frametitle{Things I plan to discuss}
\begin{center}
\begin{minipage}{2.675in}
\begin{itemize}
    \item What Neovim options are.
    \item Limitations of Neovim options.
    \item How Neovim options work internally.
    \item What stands in the way of fixing these limitations.
    \item How we plan to overcome those limitations.
    \item What you can expect going forward.
\end{itemize}
\end{minipage}
\end{center}
\end{frame}
\begin{frame}[fragile]
\frametitle{What are Neovim options?}
\begin{itemize}
    \item Values that allow interfacing with internal Neovim variables.
    \item Used to customize the behavior of Neovim or achieve desired special effects, For example:
\begin{lstlisting}
set number relativenumber  " Turn on hybrid line numbers
set nohlsearch             " Disable search highlighting
\end{lstlisting}
    \item As of now, options can be either numbers, booleans or strings (foreshadowing).
\begin{lstlisting}
set wrap             " Boolean option
set scrolloff=10     " Number option
set winbar=%f\ [%M]  " String option
\end{lstlisting}
\pause
    \item They are a nightmare to deal with.
\end{itemize}
\end{frame}
\begin{frame}
\frametitle{Limitations of Neovim options}
\begin{itemize}
    \item Only support numbers, booleans or strings.
    \item More complicated data structures need to be represented as strings.
    \item Callback options can't be set to Lua functions.
    \item No clear way to indicate an unset local value for a global-local option.
    \item Multiple interfaces for setting options, e.g., \lstinline{vim.o}, \lstinline{vim.opt}.
    \item Behavior is unintuitive and confusing, especially for window-local and global-local
    options.
\end{itemize}
\end{frame}
\begin{frame}[fragile]
\frametitle{Limitations of Neovim options}
\framesubtitle{Nontrivial types are simulated using strings}
\begin{itemize}
    \item Dict-like string options.
\begin{lstlisting}
set fillchars=fold:\ ,diff:\ 
\end{lstlisting}
    \item List-like string options:
\begin{lstlisting}
set wildmode=longest,list,full 
\end{lstlisting}
    \item Function-like string options:
\begin{lstlisting}
set completefunc=v:lua.vim.lsp.omnifunc
\end{lstlisting}
\end{itemize}
\end{frame}
\begin{frame}[fragile]
\frametitle{Limitations of Neovim options}
\framesubtitle{Appending/prepending/removing from List-like/Dict-like options}
\begin{itemize}
    \item Vim script:
\begin{lstlisting}
set fillchars=fold:\ ,diff:\  " Works as you'd expect
set fillchars+=fold:x         " Fillchars is now 'fold: ,diff: ,fold:x'
set fillchars-=fold           " Does nothing, you also need the value
\end{lstlisting}
        \item Lua:
        \begin{itemize}
            \item Not possible through \lstinline{vim.o}.
            \item \lstinline{vim.opt} allows appending, prepending and removing, but has the same
                limitations as Vim script.
        \end{itemize}
\end{itemize}
\end{frame}
\begin{frame}[fragile]
\frametitle{Limitations of Neovim options}
\framesubtitle{Function-like options can't be set to Lua options}
Well, they can.... but only through \lstinline{v:lua}, which only allows global functions or
functions that are returned by some module.
\begin{lstlisting}[style=lua_style]
local function MyFunc(...)
    -- ...
end
function _G.MyOtherFunc(...)
    -- ...
end

vim.opt.completefunc = MyFunc             -- Not possible
vim.opt.completefunc = v:lua.MyOtherFunc  -- Works
\end{lstlisting}
\end{frame}
\begin{frame}[fragile]
\frametitle{Limitations of Neovim options}
\framesubtitle{No way to indicate unset local value for global-local options}
\begin{itemize}
    \item Many global-local options use arbitrary values to represent an unset local value.
    \begin{itemize}
        \item \lstinline{'undolevels'}: \lstinline{-123456}
        \item \lstinline{'autoread'}: \lstinline{-1}
    \end{itemize}
    \item Wait, isn't \lstinline{'autoread'} a boolean option?
    \begin{itemize}
        \item Vim boolean options are actually tri-states.
        \item Only global-local options need the third state (which is \lstinline{-1}).
        \item Autoread is the only global-local options.
    \end{itemize}
    \item There is no way to unset \lstinline{'winbar'} for a single window.
    \begin{itemize}
        \item String options use empty string to indicate unset value.
        \item But an empty string can mean no winbar or an empty winbar.
    \end{itemize}
\end{itemize}
\end{frame}
\begin{frame}
\frametitle{Limitations of Neovim options}
\framesubtitle{Multiple interfaces for getting/setting an option}
\begin{itemize}
    \item Vim script:
    \begin{itemize}
        \item \lstinline{set}, \lstinline{setlocal}, \lstinline{setglobal}
        \item \lstinline{let}: \lstinline{&opt = val}, \lstinline{&b:opt = val},
        \lstinline{&g:opt = val}
        \item Vim script functions: \lstinline{getbufvar()}, \lstinline{getwinvar()},
        \lstinline{setbufvar()}, \lstinline{setwinvar()}
        \item API functions: \lstinline{nvim_get_option_value()},
        \lstinline{nvim_set_option_value()}
        \begin{itemize}
            \item Deprecated API functions: \lstinline{nvim_get_option()},
            \lstinline{nvim_set_option()}
        \end{itemize}
    \end{itemize}
    \item Lua:
    \begin{itemize}
        \item \lstinline{vim.o}, \lstinline{vim.bo}, \lstinline{vim.wo}, \lstinline{vim.go}
        \item \lstinline{vim.opt}, \lstinline{vim.opt_local}, \lstinline{vim.opt_global}
        \item API functions: same as Vim script.
    \end{itemize}
\end{itemize}
\end{frame}
\begin{frame}
\frametitle{Limitations of Neovim options}
\framesubtitle{How did all of this happen?}
\centering{I don't know for sure, but it probably went like this:}
\begin{center}
\includegraphics[keepaspectratio=true, width=3in]{images/xkcd_competing_standards.png}
\end{center}
\end{frame}
\begin{frame}
\frametitle{Limitations of Neovim options}
\framesubtitle{Option scopes: No scope for simplicity}
\begin{itemize}
    \item Options can be divided into 4 types depending on scope.
    \begin{itemize}
        \item Global: Value is global. There is no catch. This is the only scope that makes any sense.
        \item Window-local: Value is local to window... except when it isn't.
        \item Buffer-local: Value is local to buffer (... or is it?).
        \item Global-local: These options have both local and global values.
    \end{itemize}
    \item This would all be fine, except 3 out of 4 of these scopes have a catch.
    \begin{itemize}
        \item Window-local options actually have two values, one for all buffers and one for the
            curent buffer.
        \item Buffer-local options have global values too.
        \item When you set the global value for a global-local option, it also sets the local value.
    \end{itemize}
\end{itemize}
\end{frame}
\begin{frame}
\frametitle{Why is this important?}
\begin{itemize} 
    \item Options are the core component for configuring Neovim.
    \item Basically all Neovim configs have to deal with Neovim options.
    \item Every Neovim feature is connected to options in some way.
    \item Improving the behavior of options will also make it easier to add features which require
        more sophisticated configuration.
\end{itemize}
\end{frame}

\lstset{style=c_style}

\begin{frame}[fragile]
\frametitle{How Neovim options work internally}
\framesubtitle{More specifically, how they used to work before the refactor}
\pause
\begin{itemize}
    \item They don't.\pause
    \item Each option type is stored internally using a C variable type.
    \begin{itemize}
        \item Boolean options use \lstinline{int}.
        \item Number options use \lstinline{long}.
        \item String options use C-strings (\lstinline{const char *}).
    \end{itemize}
    \item Separate function for setting each option type: \lstinline{set_bool_option()},
    \lstinline{set_num_option()}, \lstinline{set_string_option()}
\end{itemize}
\end{frame}
\begin{frame}[fragile]
\frametitle{How Neovim options work internally}
\framesubtitle{More specifically, how they used to work before the refactor}
\begin{itemize}
    \item Separate code paths for each option interface.
    \begin{itemize}
        \item \lstinline[style=vim_style]{set} uses \lstinline{do_set_option_value()}.
        \item \lstinline[style=vim_style]{let} uses \lstinline{ex_let_option()}.
        \item API functions use \lstinline{set_option_value()}.
    \end{itemize}
    \item List-like, Dict-like and Function-like options are taken as strings and then manually
        processed and stored in a struct corresponding to the option.
\end{itemize}
\end{frame}
\begin{frame}[fragile]
\frametitle{Why is this so hard to fix?}
\framesubtitle{One reason: Neovim options code is a mess}
\begin{itemize}
    \item Lots of code duplication around many places (remember the separate option interfaces
        mentioned earlier?).
    \item No unified way of representing an option value.
    \item Messy, borderline unreadable code. Take a guess what this line does:
\begin{lstlisting}
*opt_idx = (((*opt)->fullname[0] == 'v')
        ? (shada_idx == -1 ? ((shada_idx = findoption("shada"))) : shada_idx)
        : *opt_idx);
\end{lstlisting}
\end{itemize}
\end{frame}
\begin{frame}
\frametitle{So.. what are we going to do about it?}
Right now, the plan is to aggressively refactor the Neovim codebase in order to:
\begin{itemize}
    \item Make the code more readable.
    \item Make options more maintainable and extensible.
    \item Reduce code duplication.
    \item Get rid of multiple interfaces for doing similar things.
    \item Remove arbitrary behavior of options (while keeping breaking changes minimal).
\end{itemize}
\end{frame}
\begin{frame}
\frametitle{What have we done so far?}
\begin{itemize}
    \item Introduced the \lstinline{OptVal} type into the codebase to represent of option value of
    any type.
    \item Unified the option set function for all option types.
    \item Unified (almost) all of the different codepaths for getting and setting an option.
    \item Many other miscellaneous improvements.\\For more info, see:
        https://github.com/neovim/neovim/issues/25672.
\end{itemize}
All of this was not done just by me. Special thanks to Lewis as well for doing a lot of the work on
refactoring options. Also thanks to Bfredl, Zeertzjq and Justin for taking their time to review PRs.
\end{frame}
\begin{frame}
\frametitle{What we're planning to do}
\framesubtitle{And what you can expect in the future}
\begin{itemize}
    \item Unify the remaining duplicate interfaces for setting options.
    \item Remove code specific to string options, to make sure that there is symmetry between all
        option types.
    \item Allow options to have multiple types.
    \item Add a unique option type that indicates an unset local value for global-local options.
    \item Add Dictionary, List and Function option types, instead of emulating those types using
        Strings.
    \item Make boolean options actually boolean.
\end{itemize}
\end{frame}
\begin{frame}
\frametitle{What we're planning to do}
\framesubtitle{And what you can expect in the future}
\begin{itemize}
    \item Changes specific to certain options:
    \begin{itemize}
        \item Allow unsetting \lstinline{'winbar'}.
        \item Add a dict-based configuration format for complicated options such as
            \lstinline{'winbar'}, \lstinline{'tabline'}, \lstinline{'statusline'}, etc.
    \end{itemize} 
\end{itemize}
\end{frame}
\begin{frame}
\frametitle{Summary}
\begin{itemize}
    \item Neovim options have a bunch of limitations, most of them related to them being restricted
        to 3 option types.
    \item Those limitations cannot yet be fixed because the Neovim options codebase is currently
        extremely difficult to maintain and extend.
    \item There has been ongoing progress to aggressively refactor Neovim options, which should
        allow us to eventually overcome these limitations.
    \item In the future, we expect to have:
    \begin{itemize}
       \item More option types
       \item Fewer interfaces
       \item More intuitive behavior
    \end{itemize}
\end{itemize}
\end{frame}
\begin{frame}
\frametitle{One last thing}
Reading Neovim options code is traumatizing.\\
Go to one of these places and donate if you want to sponsor my therapy:
\begin{itemize}
    \item https://github.com/sponsors/neovim
    \item https://opencollective.com/neovim
\end{itemize}
\end{frame}
\begin{frame}
\frametitle{Thank you!}
\begin{itemize}
    \item Thanks to the Neovim core team and other Neovim contributors for being so helpful and
        welcoming to me.
    \item Thanks to everyone who donated to Neovim and made it possible to work full-time on a
        project I love.
    \item And lastly, thank you, for spending your time listening to my talk.
\end{itemize}
\end{frame}
\end{document}
